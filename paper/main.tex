% Options for packages loaded elsewhere
\PassOptionsToPackage{unicode}{hyperref}
\PassOptionsToPackage{hyphens}{url}
\PassOptionsToPackage{dvipsnames,svgnames,x11names}{xcolor}
%
\documentclass[
  letterpaper,
  DIV=11,
  numbers=noendperiod]{scrartcl}

\usepackage{amsmath,amssymb}
\usepackage{iftex}
\ifPDFTeX
  \usepackage[T1]{fontenc}
  \usepackage[utf8]{inputenc}
  \usepackage{textcomp} % provide euro and other symbols
\else % if luatex or xetex
  \usepackage{unicode-math}
  \defaultfontfeatures{Scale=MatchLowercase}
  \defaultfontfeatures[\rmfamily]{Ligatures=TeX,Scale=1}
\fi
\usepackage{lmodern}
\ifPDFTeX\else
    % xetex/luatex font selection
\fi
% Use upquote if available, for straight quotes in verbatim environments
\IfFileExists{upquote.sty}{\usepackage{upquote}}{}
\IfFileExists{microtype.sty}{% use microtype if available
  \usepackage[]{microtype}
  \UseMicrotypeSet[protrusion]{basicmath} % disable protrusion for tt fonts
}{}
\makeatletter
\@ifundefined{KOMAClassName}{% if non-KOMA class
  \IfFileExists{parskip.sty}{%
    \usepackage{parskip}
  }{% else
    \setlength{\parindent}{0pt}
    \setlength{\parskip}{6pt plus 2pt minus 1pt}}
}{% if KOMA class
  \KOMAoptions{parskip=half}}
\makeatother
\usepackage{xcolor}
\setlength{\emergencystretch}{3em} % prevent overfull lines
\setcounter{secnumdepth}{5}
% Make \paragraph and \subparagraph free-standing
\ifx\paragraph\undefined\else
  \let\oldparagraph\paragraph
  \renewcommand{\paragraph}[1]{\oldparagraph{#1}\mbox{}}
\fi
\ifx\subparagraph\undefined\else
  \let\oldsubparagraph\subparagraph
  \renewcommand{\subparagraph}[1]{\oldsubparagraph{#1}\mbox{}}
\fi


\providecommand{\tightlist}{%
  \setlength{\itemsep}{0pt}\setlength{\parskip}{0pt}}\usepackage{longtable,booktabs,array}
\usepackage{calc} % for calculating minipage widths
% Correct order of tables after \paragraph or \subparagraph
\usepackage{etoolbox}
\makeatletter
\patchcmd\longtable{\par}{\if@noskipsec\mbox{}\fi\par}{}{}
\makeatother
% Allow footnotes in longtable head/foot
\IfFileExists{footnotehyper.sty}{\usepackage{footnotehyper}}{\usepackage{footnote}}
\makesavenoteenv{longtable}
\usepackage{graphicx}
\makeatletter
\def\maxwidth{\ifdim\Gin@nat@width>\linewidth\linewidth\else\Gin@nat@width\fi}
\def\maxheight{\ifdim\Gin@nat@height>\textheight\textheight\else\Gin@nat@height\fi}
\makeatother
% Scale images if necessary, so that they will not overflow the page
% margins by default, and it is still possible to overwrite the defaults
% using explicit options in \includegraphics[width, height, ...]{}
\setkeys{Gin}{width=\maxwidth,height=\maxheight,keepaspectratio}
% Set default figure placement to htbp
\makeatletter
\def\fps@figure{htbp}
\makeatother
\newlength{\cslhangindent}
\setlength{\cslhangindent}{1.5em}
\newlength{\csllabelwidth}
\setlength{\csllabelwidth}{3em}
\newlength{\cslentryspacingunit} % times entry-spacing
\setlength{\cslentryspacingunit}{\parskip}
\newenvironment{CSLReferences}[2] % #1 hanging-ident, #2 entry spacing
 {% don't indent paragraphs
  \setlength{\parindent}{0pt}
  % turn on hanging indent if param 1 is 1
  \ifodd #1
  \let\oldpar\par
  \def\par{\hangindent=\cslhangindent\oldpar}
  \fi
  % set entry spacing
  \setlength{\parskip}{#2\cslentryspacingunit}
 }%
 {}
\usepackage{calc}
\newcommand{\CSLBlock}[1]{#1\hfill\break}
\newcommand{\CSLLeftMargin}[1]{\parbox[t]{\csllabelwidth}{#1}}
\newcommand{\CSLRightInline}[1]{\parbox[t]{\linewidth - \csllabelwidth}{#1}\break}
\newcommand{\CSLIndent}[1]{\hspace{\cslhangindent}#1}

\KOMAoption{captions}{tableheading}
\makeatletter
\makeatother
\makeatletter
\makeatother
\makeatletter
\@ifpackageloaded{caption}{}{\usepackage{caption}}
\AtBeginDocument{%
\ifdefined\contentsname
  \renewcommand*\contentsname{Table of contents}
\else
  \newcommand\contentsname{Table of contents}
\fi
\ifdefined\listfigurename
  \renewcommand*\listfigurename{List of Figures}
\else
  \newcommand\listfigurename{List of Figures}
\fi
\ifdefined\listtablename
  \renewcommand*\listtablename{List of Tables}
\else
  \newcommand\listtablename{List of Tables}
\fi
\ifdefined\figurename
  \renewcommand*\figurename{Figure}
\else
  \newcommand\figurename{Figure}
\fi
\ifdefined\tablename
  \renewcommand*\tablename{Table}
\else
  \newcommand\tablename{Table}
\fi
}
\@ifpackageloaded{float}{}{\usepackage{float}}
\floatstyle{ruled}
\@ifundefined{c@chapter}{\newfloat{codelisting}{h}{lop}}{\newfloat{codelisting}{h}{lop}[chapter]}
\floatname{codelisting}{Listing}
\newcommand*\listoflistings{\listof{codelisting}{List of Listings}}
\makeatother
\makeatletter
\@ifpackageloaded{caption}{}{\usepackage{caption}}
\@ifpackageloaded{subcaption}{}{\usepackage{subcaption}}
\makeatother
\makeatletter
\@ifpackageloaded{tcolorbox}{}{\usepackage[skins,breakable]{tcolorbox}}
\makeatother
\makeatletter
\@ifundefined{shadecolor}{\definecolor{shadecolor}{rgb}{.97, .97, .97}}
\makeatother
\makeatletter
\makeatother
\makeatletter
\makeatother
\ifLuaTeX
  \usepackage{selnolig}  % disable illegal ligatures
\fi
\IfFileExists{bookmark.sty}{\usepackage{bookmark}}{\usepackage{hyperref}}
\IfFileExists{xurl.sty}{\usepackage{xurl}}{} % add URL line breaks if available
\urlstyle{same} % disable monospaced font for URLs
\hypersetup{
  pdftitle={Bibat: Batteries-include Bayesian Analysis Template},
  pdfauthor={Teddy Groves},
  pdfkeywords={Bayesian inference,},
  colorlinks=true,
  linkcolor={blue},
  filecolor={Maroon},
  citecolor={Blue},
  urlcolor={Blue},
  pdfcreator={LaTeX via pandoc}}

\title{Bibat: Batteries-include Bayesian Analysis Template}
\author{Teddy Groves}
\date{}

\begin{document}
\maketitle
\begin{abstract}
Despite their suitability for many scientific problems and the existence
of sound theoretical and computational frameworks, adoption of Bayesian
workflow in science remains a challenge. One reason for the difficulty
is that it is difficult to write software that implements non-trivial
Bayesian workflows. We aimed to address this difficulty by developing
Bibat, a Python package providing an interactive template that targets
Bayesian statistical analysis projects. Bibat trivialises the otherwise
painful process of structuring such a project and integrating together
tools that help with individual parts of a Bayesian workflow, without
compromising flexibility, scalability, interoperability or
reproducibility. Bibat is available on the Python Package index,
documented at \url{https://bibat.readthedocs.io/} and developed at
\url{https://github.com/teddygroves/bibat/}.
\end{abstract}
\ifdefined\Shaded\renewenvironment{Shaded}{\begin{tcolorbox}[boxrule=0pt, sharp corners, enhanced, breakable, interior hidden, borderline west={3pt}{0pt}{shadecolor}, frame hidden]}{\end{tcolorbox}}\fi

Many areas of science would benefit from replacing traditional
statistical methodologies with Bayesian workflow, as described in, for
example (Gelman et al. 2020; Grinsztajn et al. 2021; Gabry et al. 2019).

There are now software tools that address most individual aspects of a
Bayesian workflow, from fetching and manipulating data to specifying,
computing, storing, inspecting and documenting statistical inferences.
However, writing a software project implementing a Bayesian workflow
remains challenging due partly to the difficulty of orchestrating and
configuring these elements.

Bibat is a Python package that addresses this difficulty by providing an
interactive template for Bayesian workflow projects.

\hypertarget{installation-and-usage}{%
\section{Installation and usage}\label{installation-and-usage}}

Bibat is installed by running the command \texttt{pip\ install\ bibat}
and then used by running the command \texttt{bibat}.

This command triggers an interactive form which prompts the user for
configuration information including project and repository name, author
name, a short description and choices of open source licence options,
documentation formats and whether to include tests and continuous
integration.

Bibat then creates a folder with the chosen repository name in the
current working directory, containing code that implements bibat's
example analysis.

\hypertarget{documentation}{%
\section{Documentation}\label{documentation}}

Bibat is documented at \url{https://github.com/teddygroves/bibat/}. The
documentation website includes instructions for getting started, a
detailed explanation of bibat's concepts, an extended vignette
illustrating intended usage, full description of the python API and
command line interface, instructions for contributing and a section
discussing accessibility considerations.

\hypertarget{software}{%
\section{Software}\label{software}}

Bibat is implemented using Python 3. Its main dependencies are
cookiecutter (Greenfeld et al. 2021), pydantic (Pydantic developers
2022) and click (Click Developers 2022). Bibat is continuously tested to
ensure that it works on the operating systems Linux, macOS and Windows.

Bibat projects use the following tools:

\begin{itemize}
\tightlist
\item
  Python 3 (Van Rossum and Drake 2009) and standard scientific Python
  packages for data manipulation
\item
  Stan (Carpenter et al. 2017) for specifying statistical models and
  performing inferences
\item
  cmdstanpy (Stan Development Team 2022) for interfacing between Python
  and Stan
\item
  arviz (Kumar et al. 2019) for storing and analysing completed
  inferences
\item
  pydantic and pandera (Niels Bantilan 2020) for validation
\item
  make (Stallman and McGrath 1991) for automation.
\item
  Sphinx (Georg Brandl and the Sphinx team 2022) and Quarto (Allaire et
  al. 2022) for documentation.
\end{itemize}

\hypertarget{how-bibat-addresses-specific-bayesian-workflow-difficulties}{%
\section{How bibat addresses specific Bayesian workflow
difficulties}\label{how-bibat-addresses-specific-bayesian-workflow-difficulties}}

This section describes some specific problems that often affect Bayesian
workflow projects and which bibat helps to ameliorate.

\hypertarget{data-modelling}{%
\subsection{Data modelling}\label{data-modelling}}

Any statistical analysis requires a definition of a prepared dataset,
whether implicitly or explicitly. Ideally the definition is explicit and
allows for arbitrarily many prepared datasets, accommodates both tabular
and non-tabular formats and provides functionality like validation and
serialisation.

Bibat achieves this goal by providing abstract models for prepared data,
preconfigured to work well together, integrated into a data preparation
pipeline and able to be easily customised to suit any analysis.

Without a template such as bibat, users must either devise a new data
modelling framework from scratch as part of an analysis, which is
time-consuming, or else to miss out on either the flexibility or
features that bibat's approach provides.

\hypertarget{reproducibility}{%
\subsection{Reproducibility}\label{reproducibility}}

Bibat ensures reproducibility by providing a preconfigured makefile with
a target \texttt{analysis} triggering creation of an isolated
environment, installation of dependencies, data preparation, statistical
computation and analysis of results. In this way a bibat analysis can be
reproduced on most platforms using a single command.

Bibat also provides its Python code in the form of a package configured
using modern conventions for specifying dependencies and configuring
tooling, so that it is easy to maintain reproducibility as the analysis
develops.

\hypertarget{collaboration}{%
\subsection{Collaboration}\label{collaboration}}

Bibat provides a preconfigured test environment, continuous integration,
linting and pre-commit hooks, making it suitable for collaborative
software development. In addition, including documentation as a first
class component of the analysis addresses a common problem in academic
statistics projects where the paper gets out of sync with the code.

\hypertarget{installing-dependencies}{%
\subsection{Installing dependencies}\label{installing-dependencies}}

Bibat's makefile detects the current operating system and attempts to
install cmdstan appropriately if necessary. This functionality addresses
a common issue where researchers find it difficult to install Stan,
especially on Windows.

\hypertarget{fitting-modes}{%
\subsection{Fitting modes}\label{fitting-modes}}

As part of a Bayesian workflow it is often necessary to fit a model and
dataset in different ways. For example, one might perform MCMC sampling
of both the prior and posterior distributions, perform multiple
leave-out-one-fold fits for cross-validation or need to compare MCMC
sampling with an optimisation-based alternative.

Bibat accommodates this ubiquitous scenario by introducing an
abstraction called ``fitting mode'' and a corresponding Pydantic base
class \texttt{FittingMode}, along with and several subclasses for
commonly used cases.

This abstraction allows bibat projects to handle fitting a model and
dataset in different ways appropriately and flexibly. For example, the
provided prior sampling fitting mode creates a Stan input dictionary
with the \texttt{likelihood} data variable set to \texttt{0}, performs
MCMC sampling and writes data to the \texttt{InferenceData} group
\texttt{prior}.

\hypertarget{configure-fitting-concisely.}{%
\subsection{Configure fitting
concisely.}\label{configure-fitting-concisely.}}

Bibat provides a Pydantic class \texttt{InferenceConfiguration}, parsed
from toml files in the \texttt{inferences} directory. By writing a toml
file per inference the user can configure statistical model, dataset,
fitting modes and computational settings (globally or per fitting mode).

This approach appropriately separates configuration from logic and
avoids unnecessary duplication of configuration for runs with the same
model and dataset.

\hypertarget{storing-mcmc-samples}{%
\subsection{Storing MCMC samples}\label{storing-mcmc-samples}}

Bibat provides code that automatically saves its output
\texttt{InferenceData} objects in zarr format. This format distributes
information from the same set of samples into multiple smaller files,
thereby helping to avoid file size limits of online repository hosting
services.

\hypertarget{comparison-with-alternative-software}{%
\section{Comparison with alternative
software}\label{comparison-with-alternative-software}}

Other than bibat, there is currently no interactive template that
specifically targets Bayesian workflow projects. There are some
templates that arguably encompass Bayesian workflow as a special case of
data analysis project, such as cookiecutter-data-science (Driven Data
2022), but these are of limited use compared with a specialised template
due to the many specificities of Bayesian workflow.

There is some software that addresses the general task of facilitating
Bayesian workflow, but differently from bibat. For example, bambi
(Capretto et al. 2020) and brms (Bürkner 2017) aim to make implementing
Bayesian workflows easier by trivialising the task of specifying a
Bayesian generalised linear model and fitting it to a single tabular
dataset. This approach does not address several difficulties that bibat
does address,

\hypertarget{references}{%
\section*{References}\label{references}}
\addcontentsline{toc}{section}{References}

\hypertarget{refs}{}
\begin{CSLReferences}{1}{0}
\leavevmode\vadjust pre{\hypertarget{ref-Allaire_Quarto_2022}{}}%
Allaire, J. J., Charles Teague, Carlos Scheidegger, Yihui Xie, and
Christophe Dervieux. 2022. {``Quarto.''}
\url{https://doi.org/10.5281/zenodo.5960048}.

\leavevmode\vadjust pre{\hypertarget{ref-burknerBrmsPackageBayesian2017}{}}%
Bürkner, Paul-Christian. 2017. {``Brms: {An R} Package for {Bayesian}
Multilevel Models Using {Stan}.''} \emph{Journal of Statistical
Software} 80 (1): 1--28. \url{https://doi.org/10.18637/jss.v080.i01}.

\leavevmode\vadjust pre{\hypertarget{ref-capretto2020}{}}%
Capretto, Tomás, Camen Piho, Ravin Kumar, Jacob Westfall, Tal Yarkoni,
and Osvaldo A. Martin. 2020. {``Bambi: {A} Simple Interface for Fitting
{Bayesian} Linear Models in {Python}.''}
\url{https://arxiv.org/abs/2012.10754}.

\leavevmode\vadjust pre{\hypertarget{ref-carpenterStanProbabilisticProgramming2017}{}}%
Carpenter, Bob, Andrew Gelman, Matthew D. Hoffman, Daniel Lee, Ben
Goodrich, Michael Betancourt, Marcus Brubaker, Jiqiang Guo, Peter Li,
and Allen Riddell. 2017. {``Stan: {A Probabilistic Programming
Language}.''} \emph{Journal of Statistical Software} 76 (1): 1--32.
\url{https://doi.org/10.18637/jss.v076.i01}.

\leavevmode\vadjust pre{\hypertarget{ref-clickdevelopersClickPythonComposable2022}{}}%
Click Developers. 2022. {``Click: {Python} Composable Command Line
Interface Toolkit.''} Pallets. \url{https://pypi.org/project/click/}.

\leavevmode\vadjust pre{\hypertarget{ref-drivendataCookiecutterdatascience2022}{}}%
Driven Data. 2022. {``Cookiecutter-Data-Science.''}
\url{https://github.com/drivendata/cookiecutter-data-science/}.

\leavevmode\vadjust pre{\hypertarget{ref-gabryVisualizationBayesianWorkflow2019}{}}%
Gabry, Jonah, Daniel Simpson, Aki Vehtari, Michael Betancourt, and
Andrew Gelman. 2019. {``Visualization in {Bayesian Workflow}.''}
\emph{Journal of the Royal Statistical Society Series A: Statistics in
Society} 182 (2): 389--402. \url{https://doi.org/10.1111/rssa.12378}.

\leavevmode\vadjust pre{\hypertarget{ref-gelmanBayesianWorkflow2020}{}}%
Gelman, Andrew, Aki Vehtari, Daniel Simpson, Charles C. Margossian, Bob
Carpenter, Yuling Yao, Lauren Kennedy, Jonah Gabry, Paul-Christian
Bürkner, and Martin Modrák. 2020. {``Bayesian {Workflow}.''}
\emph{arXiv:2011.01808 {[}Stat{]}}, November.
\url{http://arxiv.org/abs/2011.01808}.

\leavevmode\vadjust pre{\hypertarget{ref-georgbrandlandthesphinxteamSphinx2022}{}}%
Georg Brandl and the Sphinx team. 2022. {``Sphinx.''}
\url{https://www.sphinx-doc.org/}.

\leavevmode\vadjust pre{\hypertarget{ref-greenfeldCookiecutter2021}{}}%
Greenfeld, Audrey Roy, Dainiel Roy Greenfeld, Raphael Pierzina, et al.
2021. {``Cookiecutter.''} \url{https://pypi.org/project/cookiecutter/}.

\leavevmode\vadjust pre{\hypertarget{ref-grinsztajnBayesianWorkflowDisease2021}{}}%
Grinsztajn, Léo, Elizaveta Semenova, Charles C. Margossian, and Julien
Riou. 2021. {``Bayesian Workflow for Disease Transmission Modeling in
{Stan}.''} \emph{Statistics in Medicine} 40 (27): 6209--34.
\url{https://doi.org/10.1002/sim.9164}.

\leavevmode\vadjust pre{\hypertarget{ref-kumarArviZUnifiedLibrary2019}{}}%
Kumar, Ravin, Colin Carroll, Ari Hartikainen, and Osvaldo Martin. 2019.
{``{ArviZ} a Unified Library for Exploratory Analysis of {Bayesian}
Models in {Python}.''} \emph{Journal of Open Source Software} 4 (33):
1143. \url{https://doi.org/10.21105/joss.01143}.

\leavevmode\vadjust pre{\hypertarget{ref-niels_bantilan-proc-scipy-2020}{}}%
Niels Bantilan. 2020. {``Pandera: {Statistical Data Validation} of
{Pandas Dataframes}.''} In \emph{Proceedings of the 19th {Python} in
{Science Conference}}, edited by Meghann Agarwal, Chris Calloway, Dillon
Niederhut, and David Shupe, 116--24.
\url{https://doi.org/10.25080/Majora-342d178e-010}.

\leavevmode\vadjust pre{\hypertarget{ref-pydanticdevelopersPydantic2022}{}}%
Pydantic developers. 2022. {``Pydantic.''}
\url{https://pypi.org/project/pydantic/}.

\leavevmode\vadjust pre{\hypertarget{ref-stallman1991gnu}{}}%
Stallman, Richard M, and Roland McGrath. 1991. {``{GNU} Make-{A} Program
for Directing Recompilation.''}

\leavevmode\vadjust pre{\hypertarget{ref-standevelopmentteamCmdStanPy2022}{}}%
Stan Development Team. 2022. {``{CmdStanPy}.''}
\url{https://github.com/stan-dev/cmdstanpy}.

\leavevmode\vadjust pre{\hypertarget{ref-vanrossumPythonReferenceManual2009}{}}%
Van Rossum, Guido, and Fred L. Drake. 2009. \emph{Python 3 {Reference
Manual}}. {Scotts Valley, CA}: {CreateSpace}.

\end{CSLReferences}



\end{document}
